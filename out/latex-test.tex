%\documentclass[titlepage]{article}
\documentclass{article}
\usepackage[left=2cm,right=2cm,top=2cm,bottom=2cm]{geometry}
\usepackage{graphicx}
\usepackage[spanish]{babel}
\setcounter{tocdepth}{3}
\setcounter{secnumdepth}{3}

\usepackage{titlesec}
\titleformat{\section}[block]
  {\Large\bfseries\filcenter}
  {}
  {1em}
  {}
\newcommand{\nombre}{Renato Flores}
\newcommand{\carnet}{201709244}
\newcommand{\universidad}{Universidad de San Carlos de Guatemala}
\newcommand{\catedratico}{Ing. Jose Squimux}
\newcommand{\curso}{Matematica Aplicada 2}
\newcommand{\titulo}{Tarea \#1}
\newcommand*\rbreak{\par\noindent\linebreak}

%Header :
\author{\nombre , \carnet}
\title{\textbf{\Huge\titulo}}

\begin{document}
%Use the one bellow If you want a full page caratula.
%\input{/home/renato/latex/caratula/caratula.tex}
%\clearpage

\maketitle


\section{Ejercicio 1}
\subsection{Datos}
\begin{itemize}
\item r = 5
\item c1 = 2
\item c2 = 3
\item c3 = 5
\item k = 8
\end{itemize}
\section{\Huge Cantidad Optima (Q)}
\begin{Huge}
Q = $\sqrt{2} \sqrt{\frac{c_{3} r \left(c_{1} + c_{2}\right)}{c_{1} c_{2} \left(1 - \frac{r}{k}\right)}}$\rbreak\rbreak
Q = 10*sqrt(10)/3
\end{Huge}
\section{\Huge Cantidad Maxima en inventario (S)}
\begin{Huge}
S = $\sqrt{2} \sqrt{\frac{c_{2} c_{3} r \left(1 - \frac{r}{k}\right)}{c_{1} \left(c_{1} + c_{2}\right)}}$\rbreak\rbreak
S = 3*sqrt(10)/4
\end{Huge}
\section{\Huge Costo (C)}
\begin{Huge}
C = $\sqrt{2} \sqrt{\frac{c_{1} c_{2} c_{3} r \left(1 - \frac{r}{k}\right)}{c_{1} + c_{2}}}$\rbreak\rbreak
C = 3*sqrt(10)/2
\end{Huge}
\section{\Huge Cantidad maxima de faltantes (D)}
\begin{Huge}
D = $Q - S$\rbreak\rbreak
D = 31*sqrt(10)/12
\end{Huge}
\section{\Huge tiempo 1 (t1)}
\begin{Huge}
t1 = $\sqrt{\frac{r t_{2}}{k - r}}$\rbreak\rbreak
t1 = sqrt(15)*sqrt(t2)/3
\end{Huge}
\section{\Huge tiempo 2 (t2)}
\begin{Huge}
t2 = $\sqrt{2} \sqrt{\frac{c_{2} c_{3} \left(1 - \frac{r}{k}\right)}{c_{1} r \left(c_{1} + c_{2}\right)}}$\rbreak\rbreak
t2 = 3*sqrt(10)/20
\end{Huge}
\section{\Huge tiempo 3 (t3)}
\begin{Huge}
t3 = $\sqrt{2} \sqrt{\frac{c_{1} c_{3} \left(1 - \frac{r}{k}\right)}{c_{2} r \left(c_{1} + c_{2}\right)}}$\rbreak\rbreak
t3 = sqrt(10)/10
\end{Huge}
\section{\Huge tiempo 4 (t4)}
\begin{Huge}
t4 = $\sqrt{\frac{r t_{3}}{k - r}}$\rbreak\rbreak
t4 = 2**(3/4)*sqrt(3)*5**(1/4)/6
\end{Huge}

\end{document}