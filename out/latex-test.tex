%\documentclass[titlepage]{article}
\documentclass{article}
\usepackage[left=2cm,right=2cm,top=2cm,bottom=2cm]{geometry}
\usepackage{graphicx}
\usepackage[spanish]{babel}
\setcounter{tocdepth}{3}
\setcounter{secnumdepth}{3}

\usepackage{titlesec}
\titleformat{\section}[block]
  {\Large\bfseries\filcenter}
  {}
  {1em}
  {}
\newcommand{\nombre}{Renato Flores}
\newcommand{\carnet}{201709244}
\newcommand{\universidad}{Universidad de San Carlos de Guatemala}
\newcommand{\catedratico}{Ing. Jose Squimux}
\newcommand{\curso}{Matematica Aplicada 2}
\newcommand{\titulo}{Tarea \#1}
\newcommand*\rbreak{\par\noindent\linebreak}

%Header :
\author{\nombre , \carnet}
\title{\textbf{\Huge\titulo}}

\begin{document}
%Use the one bellow If you want a full page caratula.
%\input{/home/renato/latex/caratula/caratula.tex}
%\clearpage

\maketitle


\section{\Huge Ejercicio 1}
\subsection{\Huge Datos}
\begin{huge}
\begin{itemize}
\item r = 5
\item c1 = 2
\item c2 = 3
\item c3 = 5
\item k = 8
\end{itemize}
\end{huge}\section{\Huge \Huge Inventario Optimo (Q)}
\begin{huge}
$Q = \sqrt{2} \sqrt{\frac{c_{3} r \left(c_{1} + c_{2}\right)}{c_{1} c_{2} \left(1 - \frac{r}{k}\right)}}$\rbreak
$Q = \sqrt{2} \sqrt{\frac{(5)  (5)  \left((2) + (3)\right)}{(2) (3) \left(1 - \frac{ (5) }{ (8) }\right)}}$\rbreak
$Q = 10.541$
\end{huge}
\section{\Huge \Huge Inventario Maximo (S)}
\begin{huge}
$S = \sqrt{2} \sqrt{\frac{c_{2} c_{3} r \left(1 - \frac{r}{k}\right)}{c_{1} \left(c_{1} + c_{2}\right)}}$\rbreak
$S = \sqrt{2} \sqrt{\frac{(3) (5)  (5)  \left(1 - \frac{ (5) }{ (8) }\right)}{(2) \left((2) + (3)\right)}}$\rbreak
$S = 2.372$
\end{huge}
\section{\Huge \Huge Costo Normal (C)}
\begin{huge}
$C = \sqrt{2} \sqrt{\frac{c_{1} c_{2} c_{3} r \left(1 - \frac{r}{k}\right)}{c_{1} + c_{2}}}$\rbreak
$C = \sqrt{2} \sqrt{\frac{(2) (3) (5)  (5)  \left(1 - \frac{ (5) }{ (8) }\right)}{(2) + (3)}}$\rbreak
$C = 4.743$
\end{huge}
\section{\Huge \Huge Carencia Maxima (D)}
\begin{huge}
$D = Q - S$\rbreak
$D =  (10.5409)  -  (2.3717) $\rbreak
$D = 8.169$
\end{huge}
\section{\Huge \Huge Tiempo en agotarse el inventario (t2)}
\begin{huge}
$t2 = \sqrt{2} \sqrt{\frac{c_{2} c_{3} \left(1 - \frac{r}{k}\right)}{c_{1} r \left(c_{1} + c_{2}\right)}}$\rbreak
$t2 = \sqrt{2} \sqrt{\frac{(3) (5) \left(1 - \frac{ (5) }{ (8) }\right)}{(2)  (5)  \left((2) + (3)\right)}}$\rbreak
$t2 = 0.474$
\end{huge}
\section{\Huge \Huge Tiempo de produccion (t1)}
\begin{huge}
$t1 = \sqrt{\frac{r t_{2}}{k - r}}$\rbreak
$t1 = \sqrt{\frac{ (5)  (0.4743)}{ (8)  -  (5) }}$\rbreak
$t1 = 0.889$
\end{huge}
\section{\Huge \Huge Tiempo incurrido en faltantes (t3)}
\begin{huge}
$t3 = \sqrt{2} \sqrt{\frac{c_{1} c_{3} \left(1 - \frac{r}{k}\right)}{c_{2} r \left(c_{1} + c_{2}\right)}}$\rbreak
$t3 = \sqrt{2} \sqrt{\frac{(2) (5) \left(1 - \frac{ (5) }{ (8) }\right)}{(3)  (5)  \left((2) + (3)\right)}}$\rbreak
$t3 = 0.316$
\end{huge}
\section{\Huge \Huge Tiempo en recuperar los faltantes (t4)}
\begin{huge}
$t4 = \sqrt{\frac{r t_{3}}{k - r}}$\rbreak
$t4 = \sqrt{\frac{ (5)  (0.3162)}{ (8)  -  (5) }}$\rbreak
$t4 = 0.726$
\end{huge}

\end{document}