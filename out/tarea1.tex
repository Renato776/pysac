%\documentclass[titlepage]{article}
\documentclass{article}
\usepackage[left=2cm,right=2cm,top=2cm,bottom=2cm]{geometry}
\usepackage{graphicx}
\usepackage[spanish]{babel}
\setcounter{tocdepth}{3}
\setcounter{secnumdepth}{3}

\usepackage{titlesec}
\titleformat{\section}[block]
  {\Large\bfseries\filcenter}
  {}
  {1em}
  {}
\newcommand{\nombre}{Renato Flores}
\newcommand{\carnet}{201709244}
\newcommand{\universidad}{Universidad de San Carlos de Guatemala}
\newcommand{\catedratico}{Ing. Jose Squimux}
\newcommand{\curso}{Matematica Aplicada 2}
\newcommand{\titulo}{Tarea \#1}
\newcommand*\rbreak{\par\noindent\linebreak}

%Header :
\author{\nombre , \carnet}
\title{\textbf{\Huge\titulo}}

\begin{document}
%Use the one bellow If you want a full page caratula.
%\input{/home/renato/latex/caratula/caratula.tex}
%\clearpage

\maketitle


\section{\Huge Ejercicio 0}
\subsection{\Huge Datos}
\begin{huge}
\begin{itemize}
\item p = 25
\item c3 = 50
\item r = 3000
\item c1 = 0.3*p
\item prices = [((0, 39), p), ((40, 69), 0.9*p), ((70, oo), 0.85*p)]
\end{itemize}
\end{huge}\section{\Huge \Huge q in [0,39] p = 25}
\begin{huge}
$Q = \sqrt{2} \sqrt{\frac{c_{3} r}{c_{1}}}$\rbreak
$Q = \sqrt{2} \sqrt{\frac{(50)  (3000) }{(7.5000)}}$\rbreak
$Q = 200.0$
\end{huge}
\begin{huge}
$Ct = \frac{Q c_{1}}{2} + p r + \frac{c_{3} r}{Q}$\rbreak
$Ct = \frac{ (39)  (7.5000)}{2} +  (25)   (3000)  + \frac{(50)  (3000) }{ (39) }$\rbreak
$Ct = 78992.375$
\end{huge}
\section{\Huge \Huge q in [40,69] p = 22.5000}
\begin{huge}
$Q = \sqrt{2} \sqrt{\frac{c_{3} r}{c_{1}}}$\rbreak
$Q = \sqrt{2} \sqrt{\frac{(50)  (3000) }{(6.7500)}}$\rbreak
$Q = 210.818$
\end{huge}
\begin{huge}
$Ct = \frac{Q c_{1}}{2} + p r + \frac{c_{3} r}{Q}$\rbreak
$Ct = \frac{ (69)  (6.7500)}{2} +  (22.5000)   (3000)  + \frac{(50)  (3000) }{ (69) }$\rbreak
$Ct = 69906.781$
\end{huge}
\section{\Huge \Huge q in [70,oo] p = 21.2500}
\begin{huge}
$Q = \sqrt{2} \sqrt{\frac{c_{3} r}{c_{1}}}$\rbreak
$Q = \sqrt{2} \sqrt{\frac{(50)  (3000) }{(6.3750)}}$\rbreak
$Q = 216.93$
\end{huge}
\begin{huge}
$Ct = \frac{Q c_{1}}{2} + p r + \frac{c_{3} r}{Q}$\rbreak
$Ct = \frac{ (216.9304)  (6.3750)}{2} +  (21.2500)   (3000)  + \frac{(50)  (3000) }{ (216.9304) }$\rbreak
$Ct = 65132.932$
\end{huge}
\section{\Huge Resumen}
\begin{large}
\begin{center}
\begin{tabular}{ c c c c c c }
Cantidad & Descuento & Precio & q* & Q* & Costo\\
0-39 & 0\% & 25 & 200.0 & 39 & 78992.375\\
40-69 & 10.0000\% & 22.5000 & 210.8184 & 69 & 69906.7812\\
70-oo & 15.0000\% & 21.2500 & 216.9304 & 216.9304 & 65132.9317
\end{tabular}
\end{center}
\end{large}

\section{\Huge Ejercicio 1}
\Huge Ciclo productivo con faltantes permitidos\rbreak
\subsection{\Huge Datos}
\begin{huge}
\begin{itemize}
\item r = 1666.6667
\item k = 2500
\item c1 = 0.15
\item c2 = 1.6666666666666667
\item c3 = 500
\end{itemize}
\end{huge}\section{\Huge \Huge Inventario Optimo (Q)}
\begin{huge}
$Q = \sqrt{2} \sqrt{\frac{c_{3} r \left(c_{1} + c_{2}\right)}{c_{1} c_{2} \left(1 - \frac{r}{k}\right)}}$\rbreak
$Q = \sqrt{2} \sqrt{\frac{(500)  (1666.6667)  \left((0.15) + (1.6666666666666667)\right)}{(0.15) (1.6666666666666667) \left(1 - \frac{ (1666.6667) }{ (2500) }\right)}}$\rbreak
$Q = 6027.714$
\end{huge}
\section{\Huge \Huge Inventario Maximo (S)}
\begin{huge}
$S = \sqrt{2} \sqrt{\frac{c_{2} c_{3} r \left(1 - \frac{r}{k}\right)}{c_{1} \left(c_{1} + c_{2}\right)}}$\rbreak
$S = \sqrt{2} \sqrt{\frac{(1.6666666666666667) (500)  (1666.6667)  \left(1 - \frac{ (1666.6667) }{ (2500) }\right)}{(0.15) \left((0.15) + (1.6666666666666667)\right)}}$\rbreak
$S = 1843.338$
\end{huge}
\section{\Huge \Huge Costo Normal (C)}
\begin{huge}
$C = \sqrt{2} \sqrt{\frac{c_{1} c_{2} c_{3} r \left(1 - \frac{r}{k}\right)}{c_{1} + c_{2}}}$\rbreak
$C = \sqrt{2} \sqrt{\frac{(0.15) (1.6666666666666667) (500)  (1666.6667)  \left(1 - \frac{ (1666.6667) }{ (2500) }\right)}{(0.15) + (1.6666666666666667)}}$\rbreak
$C = 276.501$
\end{huge}
\section{\Huge \Huge Carencia Maxima (D)}
\begin{huge}
$D = Q - S$\rbreak
$D =  (6027.714)  -  (1843.3375) $\rbreak
$D = 4184.377$
\end{huge}
\section{\Huge \Huge Tiempo en agotarse el inventario (t2)}
\begin{huge}
$t2 = \sqrt{2} \sqrt{\frac{c_{2} c_{3} \left(1 - \frac{r}{k}\right)}{c_{1} r \left(c_{1} + c_{2}\right)}}$\rbreak
$t2 = \sqrt{2} \sqrt{\frac{(1.6666666666666667) (500) \left(1 - \frac{ (1666.6667) }{ (2500) }\right)}{(0.15)  (1666.6667)  \left((0.15) + (1.6666666666666667)\right)}}$\rbreak
$t2 = 1.106$
\end{huge}
\section{\Huge \Huge Tiempo de produccion (t1)}
\begin{huge}
$t1 = \sqrt{\frac{r t_{2}}{k - r}}$\rbreak
$t1 = \sqrt{\frac{ (1666.6667)  (1.106)}{ (2500)  -  (1666.6667) }}$\rbreak
$t1 = 1.487$
\end{huge}
\section{\Huge \Huge Tiempo incurrido en faltantes (t3)}
\begin{huge}
$t3 = \sqrt{2} \sqrt{\frac{c_{1} c_{3} \left(1 - \frac{r}{k}\right)}{c_{2} r \left(c_{1} + c_{2}\right)}}$\rbreak
$t3 = \sqrt{2} \sqrt{\frac{(0.15) (500) \left(1 - \frac{ (1666.6667) }{ (2500) }\right)}{(1.6666666666666667)  (1666.6667)  \left((0.15) + (1.6666666666666667)\right)}}$\rbreak
$t3 = 0.1$
\end{huge}
\section{\Huge \Huge Tiempo en recuperar los faltantes (t4)}
\begin{huge}
$t4 = \sqrt{\frac{r t_{3}}{k - r}}$\rbreak
$t4 = \sqrt{\frac{ (1666.6667)  (0.0995)}{ (2500)  -  (1666.6667) }}$\rbreak
$t4 = 0.446$
\end{huge}
\section{\Huge \Huge Tiempo de produccion (tp)}
\begin{huge}
$tp = \frac{Q}{k}$\rbreak
$tp = \frac{ (6027.714) }{ (2500) }$\rbreak
$tp = 2.411$
\end{huge}
\section{\Huge \Huge tiempo de consumo (tc)}
\begin{huge}
$tc = \frac{Q}{r}$\rbreak
$tc = \frac{ (6027.714) }{ (1666.6667) }$\rbreak
$tc = 3.617$
\end{huge}
\section{\Huge \Huge Tiempo total (tt)}
\begin{huge}
$tt = tc + tp$\rbreak
$tt =  (3.6166)  +  (2.4111) $\rbreak
$tt = 6.028$
\end{huge}

\section{\Huge Ejercicio 2}
\Huge Ciclo productivo sin faltantes\rbreak
\subsection{\Huge Datos}
\begin{huge}
\begin{itemize}
\item r = 600
\item k = 1440
\item c1 = 0.0042
\item c3 = 750
\end{itemize}
\end{huge}\section{\Huge \Huge Inventario Optimo (Q)}
\begin{huge}
$Q = \sqrt{2} \sqrt{\frac{c_{3} r}{c_{1} \left(1 - \frac{r}{k}\right)}}$\rbreak
$Q = \sqrt{2} \sqrt{\frac{(750)  (600) }{(0.0042) \left(1 - \frac{ (600) }{ (1440) }\right)}}$\rbreak
$Q = 19166.297$
\end{huge}
\section{\Huge \Huge Inventario Maximo (S)}
\begin{huge}
$S = \sqrt{2} \sqrt{\frac{c_{3} r \left(1 - \frac{r}{k}\right)}{c_{1}}}$\rbreak
$S = \sqrt{2} \sqrt{\frac{(750)  (600)  \left(1 - \frac{ (600) }{ (1440) }\right)}{(0.0042)}}$\rbreak
$S = 11180.34$
\end{huge}
\section{\Huge \Huge Costo Normal (C)}
\begin{huge}
$C = \sqrt{2} \sqrt{c_{1} c_{3} r \left(1 - \frac{r}{k}\right)}$\rbreak
$C = \sqrt{2} \sqrt{(0.0042) (750)  (600)  \left(1 - \frac{ (600) }{ (1440) }\right)}$\rbreak
$C = 46.957$
\end{huge}
\section{\Huge \Huge Tiempo en agotarse el inventario (t2)}
\begin{huge}
$t2 = \sqrt{2} \sqrt{\frac{c_{3} \left(1 - \frac{r}{k}\right)}{c_{1} r}}$\rbreak
$t2 = \sqrt{2} \sqrt{\frac{(750) \left(1 - \frac{ (600) }{ (1440) }\right)}{(0.0042)  (600) }}$\rbreak
$t2 = 18.634$
\end{huge}
\section{\Huge \Huge Tiempo de produccion (tp)}
\begin{huge}
$tp = \frac{Q}{k}$\rbreak
$tp = \frac{ (19166.2969) }{ (1440) }$\rbreak
$tp = 13.31$
\end{huge}
\section{\Huge \Huge tiempo de consumo (tc)}
\begin{huge}
$tc = \frac{Q}{r}$\rbreak
$tc = \frac{ (19166.2969) }{ (600) }$\rbreak
$tc = 31.944$
\end{huge}
\section{\Huge \Huge Tiempo total (tt)}
\begin{huge}
$tt = tc + tp$\rbreak
$tt =  (31.9438)  +  (13.3099) $\rbreak
$tt = 45.254$
\end{huge}

\end{document}